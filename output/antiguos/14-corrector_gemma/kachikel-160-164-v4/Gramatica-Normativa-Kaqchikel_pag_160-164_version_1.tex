\documentclass{article}%
\usepackage[T1]{fontenc}%
\usepackage[utf8]{inputenc}%
\usepackage{lmodern}%
\usepackage{textcomp}%
\usepackage{lastpage}%
\usepackage{geometry}%
\geometry{margin=2.5cm}%
\usepackage{expex}%
\usepackage[T1]{fontenc}%
\usepackage[utf8]{inputenc}%
%
\title{Análisis Lingüístico: Gramatica{-}{-}Normativa{-}{-}Kaqchikel.pdf}%
\author{Generado por el TFM de Elías Carrasco}%
%
\begin{document}%
\normalsize%
\maketitle%
\section{Página 160}%
\label{sec:Pgina160}%
...yalan taq e koköj chik ok ri ruwäch nuya'... %
SA %
...ahora da frutos muy pequeños... %
...yalan taq e koköj chik ok ri ruwäch n{-}0{-}u{-}ya'... %
INT PL/DIM B3s pequeños PAR DIR DET A3s{-}SR INC{-}B3s{-}A3s{-}dar %
000/000/000/00 Como adverbio (modificador en predicado verbal y no verbal) %
El adjetivo funciona como adverbio en predicado verbal o no verbal. Es negado a %
través de la partícula negativa man en complemento de la partícula ta (man....ta), %
modificado por el intensificador yalan, así como modificado por las partículas na u %
ok. Ejemplos: %
Man ütz ta n{-}0{-}in{-}na' chi r{-}e la rikil la'. %
NEG bueno PAR INC{-}B3s{-}A2p{-}sentir PRE A3s{-}SR DEM comida DEM 

%
\section{Página 161}%
\label{sec:Pgina161}%

%
\section{Página 162}%
\label{sec:Pgina162}%
000/0/00 Sintagma Verbal %
La palabra principal de este tipo de sintagma es un verbo. Contiene raíz o base verbal. %
más los accidentes gramaticales, tales como el tiempo/aspecto, modo y la persona %
gramatical, como elementos obligatorios. El verbo puede tener opcionalmente afijos %
de movimiento, sufijos y partículas direccionales. Hay dos clases de sintagma verbal, %
los cuales se diferencian por el tipo de verbo: el sintagma verbal intransitivo y el sintagma %
verbal transitivo. El verbo es el primer elemento en la estructura oracional, aunque %
puede adelantársele un adverbio o algún constituyente con énfasis. %
000/0/000/0 Sintagma verbal intransitivo %
La palabra principal de este sintagma es un verbo intransitivo. Hace referencia al sujeto %
agente o paciente a través del Juego Absolutivo. La estructura del núcleo es: el tiempo %
aspecto y modo, excepto el aspecto perfectivo que va de último; movimiento (optativo), %
raíz o base verbal intransitiva y sufijos para los verbos intransitivos derivados. Ejemplos: %
Man k{-}ix{-}el el chi ruwa jay. %
NEG IMP{-}B2p{-}salir DIR PRE A3s{-}frente casa %
Ø{-}El{-}enäq el ri ixöq. %
B3s{-}Salir{-}PP DIR DET mujer 

%
\section{Página 163}%
\label{sec:Pgina163}%
Iwir x{-}0{-}il{-}itäj jun kök chi w{-}ochoch. %
ADV COM{-}B3s{-}encontrar{-}PAS IND tortuga PRE A3s.casa %
000/000/000/0/ Modificadores del verbo intransitivo %
Como queda dicho, un sintagma verbal está constituido por un verbo intransitivo más %
otros elementos que lo modifican, los que cumplen distintas funciones; algunos lo %
anteceden y otros lo posponen. Entre estos modificadores hay partículas y diferentes %
tipos de adjuntos (adverbiales o nominales). %
000/000/000/0/0/0 Partículas %
El verbo intransitivo puede ser acompañado por diversas partículas que modifican su %
significado. Las partículas pueden ser: direccionales, auxiliares, negativa más partícula %
complementaria y otras partículas como wi, kan, k'a, na, y yan. A continuación su %
descripción: 

%
\section{Página 164}%
\label{sec:Pgina164}%

%
\end{document}