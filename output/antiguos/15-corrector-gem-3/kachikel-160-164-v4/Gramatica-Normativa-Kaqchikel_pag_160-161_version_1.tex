\documentclass{article}%
\usepackage[T1]{fontenc}%
\usepackage[utf8]{inputenc}%
\usepackage{lmodern}%
\usepackage{textcomp}%
\usepackage{lastpage}%
\usepackage{geometry}%
\geometry{margin=2.5cm}%
\usepackage{expex}%
\usepackage[T1]{fontenc}%
\usepackage[utf8]{inputenc}%
%
\title{Análisis Lingüístico: Gramatica{-}{-}Normativa{-}{-}Kaqchikel.pdf}%
\author{Generado por el TFM de Elías Carrasco}%
%
\begin{document}%
\normalsize%
\maketitle%
\section{Página 160}%
\label{sec:Pgina160}%
\par\medskip%
\ex
\textit{...siembro un poco...} \\
\begingl
\gla ...ninb'än jata'q nutiko'n... //
\glb \textbf{n-}\textbf{Ø-}\textbf{in-}\textbf{bän} \textbf{jata'q} \textbf{nu-}\textbf{tiko'n} //
\glc INC-B3s-A3s-hacer poca A1s-siembra //
\glft \textbf{Sintaxis:} jata'q nutiko'n [\textsc{SA}] //
\endgl
\xe
%
\medskip%
\subsection{Como núcleo del predicado no verbal}%
\label{subsec:Comoncleodelpredicadonoverbal}%

%
Como núcleo del predicado no verbal %
El adjetivo puede ser la palabra principal de un predicado no verbal, marca al sujeto a %
través del Juego Absolutivo (elemento obligatorio). La palabra principal puede ser %
modificada por las partículas k'a (a veces en complemento con las partículas na u ok), %
la partícula negativa man en complemento de la partícula ta o por el intensificador %
yalan, estos van antes del Juego Absolutivo. Ejemplos: %
\par\medskip%
\ex
\textit{aún es mucho...} \\
\begingl
\gla ...k'a k'ïy na... //
\glb \textbf{k'a} \textbf{Ø} \textbf{k'ïy} \textbf{na} //
\glc PAR B3s mucho PAR //
\glft \textbf{Sintaxis:} k'a k'ïy na [\textsc{SA}] //
\endgl
\xe
%
\medskip%
\par\medskip%
\ex
\textit{...aún era pequeño...} \\
\begingl
\gla ...k'a in ko'öl ok ri'. //
\glb \textbf{k'a} \textbf{in} \textbf{ko'öl} \textbf{ok} \textbf{ri'} //
\glc PAR B1s pequeño DIR DEM //
\glft \textbf{Sintaxis:} k'a in ko'öl ok ri' [\textsc{SA}] //
\endgl
\xe
%
\medskip%
\par\medskip%
\ex
\textit{...aún somos pequeños...} \\
\begingl
\gla ...k'a öj koköj ok... //
\glb \textbf{k'a} \textbf{öj} \textbf{koköj} \textbf{ok} //
\glc PAR nosotros pequeños DIR //
\glft \textbf{Sintaxis:} k'a öj koköj ok [\textsc{SA}] //
\endgl
\xe
%
\medskip%
\par\medskip%
\ex
\textit{El güipil no es rojo.} \\
\begingl
\gla Man käq ta rub'onil ri potaj. //
\glb \textbf{Man} \textbf{Ø} \textbf{käq} \textbf{ta} \textbf{ru-}\textbf{b'onil} \textbf{ri} \textbf{potaj} //
\glc NEG B3s rojo PAR A3s-color DET güipil //
\glft \textbf{Sintaxis:} Man käq ta rub'onil ri potaj [\textsc{SA}] //
\endgl
\xe
%
\medskip

%
\end{document}