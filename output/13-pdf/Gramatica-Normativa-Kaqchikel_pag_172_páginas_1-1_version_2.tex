\documentclass{article}%
\usepackage[T1]{fontenc}%
\usepackage[utf8]{inputenc}%
\usepackage{lmodern}%
\usepackage{textcomp}%
\usepackage{lastpage}%
\usepackage{geometry}%
\geometry{margin=2.5cm}%
\usepackage{expex}%
\usepackage[T1]{fontenc}%
\usepackage[utf8]{inputenc}%
%
\title{Análisis Lingüístico: Gramatica{-}{-}Normativa{-}{-}Kaqchikel pag 172.pdf}%
\author{Generado por el TFM de Elías Carrasco}%
%
\begin{document}%
\normalsize%
\maketitle%
\section{Página 170}%
\label{sec:Pgina170}%
plural) del objeto y sujeto. La palabra principal incluye primeramente el tiempo/aspecto %
o modo (excepto en el perfectivo que es la última), seguido por el juego absolutivo, %
juego ergativo, la raíz o base en calidad de obligatorios y sufijos de categoría en casos %
específicos (si el verbo es derivado). Como elementos opcionales incluye el movimiento, %
su posición es siempre antes del juego ergativo. Ejemplos: %
\subsection{Marca de tiempo/aspecto/modo}%
\label{subsec:Marcadetiempo/aspecto/modo}%

%
Marca de tiempo/aspecto/modo: %
\par\medskip%
\ex
\textit{Todavía fui a ver al niño.} \\
\begingl
\gla Xb'entz'eta' na kan ri ak'wal. //
\glb \textbf{X-}\textbf{0-}\textbf{b'e-}\textbf{n-}\textbf{tz'et-}\textbf{a'} \textbf{na} \textbf{kan} \textbf{ri} \textbf{ak'wal} //
\glc COM-B3S-MOV-Als-ver-SC PAR DIR DET niño //
\endgl
\xe
%
\medskip%
\subsection{Marca el objeto en el verbo}%
\label{subsec:Marcaelobjetoenelverbo}%

%
Marca el objeto en el verbo: %
\par\medskip%
\ex
\textit{...vio muy hermosas a las niñas...} \\
\begingl
\gla ...kan jeb'ël ok xerutz'ët ri ch'uta'q xtani' rija'... //
\glb \textbf{kan} \textbf{jeb'ël} \textbf{ok} \textbf{x-}\textbf{e-}\textbf{ru-}\textbf{tz'ët} \textbf{ri} \textbf{ch'uta'q} \textbf{xtani'} \textbf{rija'} //
\glc PAR hermosa DIR COM-B3p-A3s-ver DET DIM Niña-PL él //
\endgl
\xe
%
\medskip%
\subsection{Marca el sujeto en el verbo}%
\label{subsec:Marcaelsujetoenelverbo}%

%
Marca el sujeto en el verbo: %
\par\medskip%
\ex
\textit{Los dos hombres compraron un corte.} \\
\begingl
\gla Xkilöq' jun uqaj ri ka'i' achi'a'. //
\glb \textbf{X-}\textbf{0-}\textbf{ki-}\textbf{löq} \textbf{jun} \textbf{uqaj} \textbf{ri} \textbf{ka'i'} \textbf{achi'a'.} //
\glc COM-B3s-A3p-comprar IND corte DET dos hombre-PL //
\endgl
\xe
%
\medskip%
\par\medskip%
\ex
\textit{La señorita regaló el tomate.} \\
\begingl
\gla Xusipaj ri xkoya' ri xtän. //
\glb \textbf{X-}\textbf{0-}\textbf{u-}\textbf{sip-}\textbf{a-}\textbf{j} \textbf{ri} \textbf{xkoya'} \textbf{ri} \textbf{xtän.} //
\glc COM-B3s-A3s-regalar-BV-SC DET tomate DET señorita //
\endgl
\xe
%
\medskip%
\subsection{Marca movimiento en el verbo}%
\label{subsec:Marcamovimientoenelverbo}%

%
Marca movimiento en el verbo: %
\par\medskip%
\ex
\textit{...nuestro papá va a trabajar, vamos a ayudarlo....} \\
\begingl
\gla ...nb'esamäj qatata', nb'eqato',.... //
\glb \textbf{nb'esamäj} \textbf{qatata',} \textbf{nb'eqato',} //
\glc SVT SVT SVT //
\endgl
\xe
%
\medskip

%
\end{document}