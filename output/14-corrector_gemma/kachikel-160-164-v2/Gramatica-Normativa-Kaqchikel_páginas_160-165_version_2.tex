\documentclass{article}%
\usepackage[T1]{fontenc}%
\usepackage[utf8]{inputenc}%
\usepackage{lmodern}%
\usepackage{textcomp}%
\usepackage{lastpage}%
\usepackage{geometry}%
\geometry{margin=2.5cm}%
\usepackage{expex}%
\usepackage[T1]{fontenc}%
\usepackage[utf8]{inputenc}%
%
\title{Análisis Lingüístico: Gramatica{-}{-}Normativa{-}{-}Kaqchikel.pdf}%
\author{Generado por el TFM de Elías Carrasco}%
%
\begin{document}%
\normalsize%
\maketitle%
\section{Página 160}%
\label{sec:Pgina160}%
\subsection{Sintagma Adjetival}%
\label{subsec:SintagmaAdjetival}%

%
۰۰۰/۰/۰۰ Sintagma Adjetival %
La palabra principal del sintagma adjetival es uno o más adjetivos; cuando es más de %
un adjetivo actúan de manera complementaria. Está compuesto facultativamente por la %
raíz y en casos específicos por sufijos de plural. Ejemplos: %
\par\medskip%
\ex
\textit{...encontramos graves enfermedades,...} \\
\begingl
\gla ...nqïl nima'q taq yab'il,... //
\glb \textbf{n} \textbf{Ø-}\textbf{q-}\textbf{il} \textbf{nim-a'q} \textbf{taq} \textbf{yab'il,...} //
\glc INC- B3s-A1s-encontrar grande-PL PL/DIM enfermedad //
\glft \textbf{Sintaxis:} nima'q [\textsc{SA}] //
\endgl
\xe
%
\medskip%
\par\medskip%
\ex
\textit{...así como las niñas...} \\
\begingl
\gla ...achi'el ri ch'uta'q xtani'... //
\glb \textbf{achi'el} \textbf{ri} \textbf{ch'ut-a'q} \textbf{xtan-i'...} //
\glc PAR DET pequeño-PL niña-PL //
\endgl
\xe
%
\medskip%
\subsection{Modificadores del adjetivo}%
\label{subsec:Modificadoresdeladjetivo}%

%
۰۰۰/۰۰۰/۰ Modificadores del adjetivo %
Este sintagma puede estar formado por más de un adjetivo, funcionan de manera %
complementaria, generalmente indica plural. También es modificado por las partículas %
chik y ok o por el intensificador yalan. Ejemplos: %
\par\medskip%
\ex
\textit{usaron la otra clase de ropa, las más livianas...} \\
\begingl
\gla ...xkokisaj yan qa la juley chik tzyäq, la ch'uta'q xax ok xkokisaj qa... //
\glb \textbf{x-}\textbf{Ø-}\textbf{k-}\textbf{ok-isa-j} \textbf{yan} \textbf{qa} \textbf{la} \textbf{juley} \textbf{chik} \textbf{tzyäq,} \textbf{la} \textbf{ch'ut-a'q} \textbf{xax} \textbf{xkokisaj} \textbf{qa...} //
\glc COM-B3s-A3p-usar-CAU-SC SA DEM DEM uno-clase PAR DIR DEM pequeño-PL ok SA DEM //
\endgl
\xe
%
\medskip%
\par\medskip%
\ex
\textit{usaron la otra clase de ropa, las más livianas...} \\
\begingl
\gla x-Ø-k-ok-isa-j //
\glb \textbf{x-}\textbf{Ø-}\textbf{k-}\textbf{ok-isa-j} \textbf{ch'ut-a'q} \textbf{xax} \textbf{xkokisaj} \textbf{qa...} //
\glc COM-B3s-A3p-usar-CAU-SC pequeño-PL ok SA DEM //
\endgl
\xe
%
\medskip%
\par\medskip%
\ex
\textit{...corte común fajas pequeñas comunes...} \\
\begingl
\gla ...relik uqaj, relik ok ch'uta'q pas... //
\glb \textbf{relik} \textbf{uqaj,} \textbf{relik} \textbf{ok} \textbf{ch'ut-a'q} \textbf{pas...} //
\glc SA corte SA común pequeño-PL faja //
\endgl
\xe
%
\medskip

%
\section{Página 161}%
\label{sec:Pgina161}%
\subsection{Funciones del Sintagma Adjetival}%
\label{subsec:FuncionesdelSintagmaAdjetival}%

%
...relik uqaj, relik ok ch'ut{-}a'q pas... %
...corte común fajas pequeñas comunes... %
Funciones del Sintagma Adjetival %
El sintagma adjetival funciona dentro de la cláusula así: %
۰۰۰/۰/۰۰۰۰/۰ %
Como modificador del sintagma nominal. %
Como núcleo del predicado no verbal. %
Como adverbio (modificador en predicados verbal y no verbal). %
A continuación se detalla cada una de las funciones del sintagma adjetival y los %
modificadores para cada caso: %
۰۰۰/۰/۰۰۰۰/۰ %
Como modificador del sintagma nominal %
El adjetivo se antepone al nominal para modificarlo, los adjetivos siempre concuerdan %
en cuanto al número, es decir, si el nominal es plural, el adjetivo también se escribirá %
en plural. Ejemplos: %
\par\medskip%
\ex
\textit{...mujeres importantes....} \\
\begingl
\gla ...nima'q taq ixoqi'... //
\glb \textbf{nima'q} \textbf{taq} \textbf{ixoqi'...} //
\glc SA PL/DIM mujer //
\endgl
\xe
%
\medskip%
\par\medskip%
\ex
\textit{grande-PL mujer-PL} \\
\begingl
\gla ...nim-a'q taq ixoq-i'... //
\glb \textbf{nim-a'q} \textbf{taq} \textbf{ixoq-i'...} //
\glc grande-PL PL/DIM mujer-PL //
\endgl
\xe
%
\medskip%
\par\medskip%
\ex
\textit{En ese pueblo hay casas muy grandes.} \\
\begingl
\gla ...chi rupam la jun tinamït la' k'o mama' taq jay... //
\glb \textbf{chi} \textbf{rupam} \textbf{la} \textbf{jun} \textbf{tinamït} \textbf{la'} \textbf{k'o} \textbf{mama'} \textbf{taq} \textbf{jay...} //
\glc PRE A3s-PRE DEM IND pueblo DEM B3s grande PL/DIM casa //
\endgl
\xe
%
\medskip%
\par\medskip%
\ex
\textit{...pájaros pequeños, negros....} \\
\begingl
\gla ...ch'uta'q koköj ok chikopi', xaq ok kij.... //
\glb \textbf{ch'ut-a'q} \textbf{koköj} \textbf{ok} \textbf{chikopi',} \textbf{xaq} \textbf{ok} \textbf{kij....} //
\glc pequeño-PL pequeño DIR pájaro negro DIR A3p-espalda //
\endgl
\xe
%
\medskip

%
\section{Página 162}%
\label{sec:Pgina162}%
\subsection{Sintagma Verbal}%
\label{subsec:SintagmaVerbal}%

%
...xa choj xowär... %
...solamente vino a dormir... %
\par\medskip%
\ex
\textit{...solamente vino a dormir...} \\
\begingl
\gla xa choj x-0-o-wär... //
\glb \textbf{xa} \textbf{choj} \textbf{x-}\textbf{0-}\textbf{o-}\textbf{wär...} //
\glc ADV PAR COM-B3s-MOV-dormir //
\endgl
\xe
%
\medskip%
...toq xoqa chi rochoch.... %
...cuando vino a su casa,... %
\par\medskip%
\ex
\textit{...cuando vino a su casa,...} \\
\begingl
\gla ...toq xoqa chi rochoch,... //
\glb \textbf{toq} \textbf{x-}\textbf{0-}\textbf{o-}\textbf{qa} \textbf{chi} \textbf{r-ochoch,...} //
\glc Cuando COM-B3s-MOV-vino PRE A3s-casa //
\endgl
\xe
%
\medskip%
\par\medskip%
\ex
\textit{...se vino a hincar y pidió perdon.} \\
\begingl
\gla xoxuke', xuk'utuj rukuyb'äl mak. //
\glb \textbf{x-}\textbf{4-}\textbf{o-}\textbf{xuk-e',} \textbf{x-}\textbf{0-}\textbf{u-}\textbf{kutuj} \textbf{ru-}\textbf{kuy-}\textbf{b'äl} \textbf{mak.} //
\glc COM-MOV-hincar-SC COM-B3s-A3s-pedir A3s-perdonar-INTR culpa //
\endgl
\xe
%
\medskip%
\par\medskip%
\ex
\textit{Las vacas se terminaron....} \\
\begingl
\gla Xek'is el ri wakx.... //
\glb \textbf{Xek'is} \textbf{el} \textbf{ri} \textbf{wakx,...} //
\glc SVI el DET vaca //
\endgl
\xe
%
\medskip%
\par\medskip%
\ex
\textit{Las vacas se terminaron....} \\
\begingl
\gla X-e-k'is el ri wakx,... //
\glb \textbf{X-}\textbf{e-}\textbf{k'is} \textbf{el} \textbf{ri} \textbf{wakx,...} //
\glc COM-B3p-terminar DIR DET vaca //
\endgl
\xe
%
\medskip%
\par\medskip%
\ex
\textit{Ayer se encontró una tortuga en mi casa.} \\
\begingl
\gla Iwir xilitäj jun kök chi wochoch //
\glb \textbf{Iwir} \textbf{xilitäj} \textbf{jun} \textbf{kök} \textbf{chi} \textbf{wochoch} //
\glc SVI encontró una tortuga en mi casa //
\endgl
\xe
%
\medskip%
\par\medskip%
\ex
\textit{Ayer se encontró una tortuga en mi casa.} \\
\begingl
\gla iwir x-0-il-itäj //
\glb \textbf{iwir} \textbf{x-}\textbf{0-}\textbf{il-}\textbf{itäj} \textbf{jun} \textbf{kök} \textbf{chi} \textbf{w-ochoch.} //
\glc ADV COM-B3s-encontrar-PAS IND tortuga PRE A3s.casa //
\endgl
\xe
%
\medskip

%
\section{Página 163}%
\label{sec:Pgina163}%
\subsection{Sintagma Verbal intransitivo}%
\label{subsec:SintagmaVerbalintransitivo}%

%
۰۰۰/۰/۰۰۰/۰ Sintagma verbal intransitivo %
La palabra principal de este sintagma es un verbo intransitivo. Hace referencia al sujeto %
agente o paciente a través del Juego Absolutivo. La estructura del núcleo es: el tiempo %
aspecto y modo, excepto el aspecto perfectivo que va de último; movimiento (optativo), %
raíz o base verbal intransitiva y sufijos para los verbos intransitivos derivados. Ejemplos: %
\par\medskip%
\ex
\textit{El muchacho vino a sentarse.} \\
\begingl
\gla Xpe ri ala', xotz'uye'. //
\glb \textbf{Xpe} \textbf{ri} \textbf{ala',} \textbf{x-}\textbf{0-}\textbf{o-}\textbf{pe} \textbf{x-}\textbf{0-}\textbf{o-}\textbf{tz'uy-e'.} //
\glc SVI DET muchacho COM-B3s-venir-MOV COM-B3s-MOV-sentar-SC //
\endgl
\xe
%
\medskip%
\par\medskip%
\ex
\textit{No salgan al patio.} \\
\begingl
\gla Man kixel el chi ruwa jay. //
\glb \textbf{Man} \textbf{k-ix-el} \textbf{el} \textbf{chi} \textbf{ruwa} \textbf{jay.} //
\glc SVI IMP-B2p-salir DIR PRE A3s-frente casa //
\endgl
\xe
%
\medskip%
\par\medskip%
\ex
\textit{La mujer ha salido} \\
\begingl
\gla Elenäq el ri ixöq. //
\glb \textbf{Elenäq} \textbf{el} \textbf{ri} \textbf{ixöq.} //
\glc SVI el DET mujer //
\endgl
\xe
%
\medskip%
\par\medskip%
\ex
\textit{La mujer ha salido} \\
\begingl
\gla Ø-El-enäq el ri ixöq. //
\glb \textbf{Ø-}\textbf{El-}\textbf{enäq} \textbf{el} \textbf{ri} \textbf{ixöq.} //
\glc B3s-Salir-PP DIR DET mujer //
\endgl
\xe
%
\medskip

%
\section{Página 164}%
\label{sec:Pgina164}%
\subsection{Modificadores del verbo intransitivo}%
\label{subsec:Modificadoresdelverbointransitivo}%

%
...xa choj xowär... %
...solamente vino a dormir... %
\par\medskip%
\ex
\textit{...solamente vino a dormir...} \\
\begingl
\gla xa choj x-0-o-wär... //
\glb \textbf{xa} \textbf{choj} \textbf{x-}\textbf{0-}\textbf{o-}\textbf{wär...} //
\glc ADV PAR COM-B3s-MOV-dormir //
\endgl
\xe
%
\medskip%
۰۰۰/۰/۰۰۰/۰/ Modificadores del verbo intransitivo %
Como queda dicho, un sintagma verbal está constituido por un verbo intransitivo más %
otros elementos que lo modifican, los que cumplen distintas funciones; algunos lo %
anteceden y otros lo posponen. Entre estos modificadores hay partículas y diferentes %
tipos de adjuntos (adverbiales o nominales). %
٠٠٠/٠/٠٠٠/٠/٠/ %
Partículas %
El verbo intransitivo puede ser acompañado por diversas partículas que modifican su %
significado. Las partículas pueden ser: direccionales, auxiliares, negativa más partícula %
complementaria y otras partículas como wi, kan, k'a, na, y yan. A continuación su %
descripción: 

%
\section{Página 165}%
\label{sec:Pgina165}%
\par\medskip%
\ex
\textit{...dormir...} \\
\begingl
\gla ...xa choj x-0-o-wär... //
\glb \textbf{xa} \textbf{choj} \textbf{x-}\textbf{0-}\textbf{o-}\textbf{wär.} //
\glc PAR ADV COM-B3s-MOV-dormir //
\endgl
\xe
%
\medskip%
\par\medskip%
\ex
\textit{...cuando vino a su casa....} \\
\begingl
\gla ...toq xoqa chi rochoch.... //
\glb \textbf{toq} \textbf{xoqa} \textbf{chi} \textbf{r-ochoch,} //
\glc SVI COM-B3s-MOV-vino PRE A3s-casa //
\endgl
\xe
%
\medskip%
\par\medskip%
\ex
\textit{Cuando} \\
\begingl
\gla ...toq //
\glb \textbf{toq} //
\glc Cuando //
\endgl
\xe
%
\medskip%
\par\medskip%
\ex
\textit{COM-B3S-MOV-vino} \\
\begingl
\gla x-0-o-qa //
\glb \textbf{x-}\textbf{0-}\textbf{o-}\textbf{qa} //
\glc COM-B3S-MOV-vino //
\endgl
\xe
%
\medskip%
\par\medskip%
\ex
\textit{PRE A3s-casa} \\
\begingl
\gla chi r-ochoch, //
\glb \textbf{chi} \textbf{r-}\textbf{ochoch,} //
\glc PRE A3s-casa //
\endgl
\xe
%
\medskip%
\par\medskip%
\ex
\textit{...se vino a hincar y pidió perdon.} \\
\begingl
\gla xoxuke', xuk'utuj rukuyb'äl mak. //
\glb \textbf{xoxuke',} \textbf{xuk'utuj} \textbf{rukuyb'äl} \textbf{mak.} //
\glc SVI COM-B3s-hincar-SC A3s-perdonar-INTR culpa //
\endgl
\xe
%
\medskip%
\par\medskip%
\ex
\textit{COM-B3s-MOV-hincar-SC} \\
\begingl
\gla x-4-o-xuk-e', //
\glb \textbf{x-}\textbf{4-}\textbf{o-}\textbf{xuk-}\textbf{e',} //
\glc COM-B3s-MOV-hincar-SC //
\endgl
\xe
%
\medskip%
\par\medskip%
\ex
\textit{COM-B3s-A3s-pedir} \\
\begingl
\gla x-0-u-kutuj //
\glb \textbf{x-}\textbf{0-}\textbf{u-}\textbf{kutuj} //
\glc COM-B3s-A3s-pedir //
\endgl
\xe
%
\medskip%
\par\medskip%
\ex
\textit{A3s-perdonar-INTR} \\
\begingl
\gla ru-kuy-b'äl //
\glb \textbf{ru-}\textbf{kuy-}\textbf{b'äl} //
\glc A3s-perdonar-INTR //
\endgl
\xe
%
\medskip%
\par\medskip%
\ex
\textit{culpa} \\
\begingl
\gla mak. //
\glb \textbf{mak.} //
\glc culpa //
\endgl
\xe
%
\medskip%
\par\medskip%
\ex
\textit{Las vacas se terminaron....} \\
\begingl
\gla Xek'is el ri wakx.... //
\glb \textbf{Xek'is} \textbf{el} \textbf{ri} \textbf{wakx....} //
\glc SVI DIR DET vaca //
\endgl
\xe
%
\medskip%
\par\medskip%
\ex
\textit{COM-B3p-terminar} \\
\begingl
\gla X-e-k'is //
\glb \textbf{X-}\textbf{e-}\textbf{k'is} //
\glc COM-B3p-terminar //
\endgl
\xe
%
\medskip%
\par\medskip%
\ex
\textit{DIR} \\
\begingl
\gla el //
\glb \textbf{el} //
\glc DIR //
\endgl
\xe
%
\medskip%
\par\medskip%
\ex
\textit{DET} \\
\begingl
\gla ri //
\glb \textbf{ri} //
\glc DET //
\endgl
\xe
%
\medskip%
\par\medskip%
\ex
\textit{vaca} \\
\begingl
\gla wakx,... //
\glb \textbf{wakx,...} //
\glc vaca //
\endgl
\xe
%
\medskip%
\par\medskip%
\ex
\textit{Ayer se encontró una tortuga en mi casa.} \\
\begingl
\gla Iwir xilitäj jun kök chi wochoch //
\glb \textbf{Iwir} \textbf{xilitäj} \textbf{jun} \textbf{kök} \textbf{chi} \textbf{w-ochoch.} //
\glc SVI COM-B3s-encontrar-PAS IND tortuga PRE A3s.casa //
\endgl
\xe
%
\medskip%
\par\medskip%
\ex
\textit{ADV} \\
\begingl
\gla iwir //
\glb \textbf{iwir} //
\glc ADV //
\endgl
\xe
%
\medskip%
\par\medskip%
\ex
\textit{COM-B3s-encontrar-PAS} \\
\begingl
\gla x-0-il-itäj //
\glb \textbf{x-}\textbf{0-}\textbf{il-}\textbf{itäj} //
\glc COM-B3s-encontrar-PAS //
\endgl
\xe
%
\medskip%
\par\medskip%
\ex
\textit{IND} \\
\begingl
\gla jun //
\glb \textbf{jun} //
\glc IND //
\endgl
\xe
%
\medskip%
\par\medskip%
\ex
\textit{tortuga} \\
\begingl
\gla kök //
\glb \textbf{kök} //
\glc tortuga //
\endgl
\xe
%
\medskip%
\par\medskip%
\ex
\textit{PRE} \\
\begingl
\gla chi //
\glb \textbf{chi} //
\glc PRE //
\endgl
\xe
%
\medskip%
\par\medskip%
\ex
\textit{A3s.casa} \\
\begingl
\gla w-ochoch. //
\glb \textbf{w-}\textbf{ochoch.} //
\glc A3s.casa //
\endgl
\xe
%
\medskip%
Como queda dicho, un sintagma verbal está compuesto por un verbo intransitivo más %
otros elementos que lo modifican, los que cumplen distintas funciones; algunos lo %
anteceden y otros lo posponen. Entre estos modificadores hay partículas y diferentes %
tipos de adjuntos (adverbiales o nominales). %
٠٠٠/٠/٠٠٠/٠/٠/ %
\subsection{Partículas}%
\label{subsec:Partculas}%

%
Partículas %
El verbo intransitivo puede ser acompañado por diversas partículas que modifican su %
significado. Las partículas pueden ser: direccionales, auxiliares, negativa más partícula %
complementaria y otras partículas como wi, kan, k'a, na, y yan. A continuación su %
descripción: 

%
\end{document}