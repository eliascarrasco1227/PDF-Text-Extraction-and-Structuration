\documentclass{article}%
\usepackage[T1]{fontenc}%
\usepackage[utf8]{inputenc}%
\usepackage{lmodern}%
\usepackage{textcomp}%
\usepackage{lastpage}%
\usepackage{geometry}%
\geometry{margin=2.5cm}%
\usepackage{expex}%
\usepackage[T1]{fontenc}%
\usepackage[utf8]{inputenc}%
%
\title{Análisis Lingüístico: Gramatica{-}{-}Normativa{-}{-}Kaqchikel.pdf}%
\author{Generado por el TFM de Elías Carrasco}%
%
\begin{document}%
\normalsize%
\maketitle%
\section{Página 158}%
\label{sec:Pgina158}%
\subsection{Seccion}%
\label{subsec:Seccion}%

%
Sintagma Adjetival %
La palabra principal del sintagma adjetival es uno o más adjetivos; cuando es más de %
un adjetivo actúan de manera complementaria. Está compuesto facultativamente por la %
raíz y en casos específicos por sufijos de plural. Ejemplos: %
\par\medskip%
\ex
\textit{...encontramos graves enfermedades,...} \\
\begingl
\gla ...nqil nima'q taq yab'il,... //
\glb \textbf{n-}\textbf{0-}\textbf{q-}\textbf{il} \textbf{nim-}\textbf{a'qtaq} \textbf{yab'il,...} //
\glc INC-B3s-A1s-encontrar grande-PL PL/DIM enfermedad //
\glft \textbf{Sintaxis:} nima'q [\textsc{SA}] //
\endgl
\xe
%
\medskip%
\par\medskip%
\ex
\textit{...así como las niñas...} \\
\begingl
\gla ...achi'el ri ch'uta'q xtani'... //
\glb \textbf{achi'el} \textbf{ri} \textbf{ch'ut-}\textbf{a'q} \textbf{xtan-}\textbf{i'...} //
\glc PAR DET pequeño-PL niña-PL //
\glft \textbf{Sintaxis:} ch'uta'q [\textsc{SA}] //
\endgl
\xe
%
\medskip%
\subsection{Seccion}%
\label{subsec:Seccion}%

%
Modificadores del adjetivo %
Este sintagma puede estar formado por más de un adjetivo, funcionan de manera %
complementaria, generalmente indica plural. También es modificado por las partículas %
chik y ok o por el intensificador yalan. Ejemplos: %
\par\medskip%
\ex
\textit{...usaron la otra clase de ropa, las más livianas...} \\
\begingl
\gla ...xkokisaj yan qa la juley chik tzyäq, la ch'uta'q xax ok xkokisaj qa... //
\glb \textbf{x-}\textbf{0-}\textbf{k-}\textbf{ok-}\textbf{isa-}\textbf{j} \textbf{yan} \textbf{qa} \textbf{la} \textbf{ju-}\textbf{ley} \textbf{chik} \textbf{tzyäq,} \textbf{la} \textbf{ch'ut-}\textbf{a'q} \textbf{xax} \textbf{ok} \textbf{x-}\textbf{0-}\textbf{k-}\textbf{ok-}\textbf{isa-}\textbf{j} \textbf{qa...} //
\glc COM-B3s-A3p-usar-CAU-SC PAR DIR DEM uno-clase PAR ropa DEM pequeño-PL delgado DIR COM-B3s-A3p-usar-CAU-SC DIR //
\glft \textbf{Sintaxis:} juley chik tzyäq [\textsc{SA}] ch'uta'q xax ok [\textsc{SA}] //
\endgl
\xe
%
\medskip%
\par\medskip%
\ex
\textit{...corte común fajas pequeñas comunes...} \\
\begingl
\gla ...relik uqaj, relik ok ch'uta'q pas... //
\glb \textbf{relik} \textbf{uqaj,} \textbf{relik} \textbf{ok} \textbf{ch'ut-}\textbf{a'q} \textbf{pas...} //
\glc común corte común DIR pequeño-PL faja //
\glft \textbf{Sintaxis:} relik [\textsc{SA}] relik ok ch'uta'q [\textsc{SA}] //
\endgl
\xe
%
\medskip

%
\section{Página 159}%
\label{sec:Pgina159}%
\subsection{Seccion}%
\label{subsec:Seccion}%

%
Funciones del Sintagma Adjetival %
El sintagma adjetival funciona dentro de la cláusula así: %
Como modificador del sintagma nominal. %
Como núcleo del predicado no verbal. %
Como adverbio (modificador en predicados verbal y no verbal). %
A continuación se detalla cada una de las funciones del sintagma adjetival y los %
modificadores para cada caso: %
\subsection{Seccion}%
\label{subsec:Seccion}%

%
Como modificador del sintagma nominal %
El adjetivo se antepone al nominal para modificarlo, los adjetivos siempre concuerdan %
en cuanto al número, es decir, si el nominal es plural, el adjetivo también se escribirá %
en plural. Ejemplos: %
\par\medskip%
\ex
\textit{...mujeres importantes,...} \\
\begingl
\gla ...nima'q taq ixoqi',... //
\glb \textbf{nima'q} //
\glc SA //
\endgl
\xe
%
\medskip%
\par\medskip%
\ex
\textit{...mujeres importantes,...} \\
\begingl
\gla ...nim-a'q taq ixoq-i',... //
\glb \textbf{nim-a'q} \textbf{taq} \textbf{ixoq-i',...} //
\glc grande-PL PL/DIM mujer-PL //
\endgl
\xe
%
\medskip

%
\end{document}